\documentclass[]{article}
\usepackage[utf8]{inputenc}
\usepackage{natbib}
\usepackage{graphicx}
\usepackage{geometry}
\usepackage{hyperref}


\geometry{
	a4paper,
	total={170mm,257mm},
	left=20mm,
	top=20mm,
}

%opening
\title{Progetto di Algoritmi Avanzati  	
		\\ \large Capacitated Vehicle Routing Problem 
		 \\  Algoritmi: Costruttivi, a 2 Fasi e Genetici}
	 

\author{Giulio Pilotto - Matricola:1140718}

\date{Luglio 2019}

\begin{document}

\maketitle

\begin{abstract}
Questo elaborato presenta il progetto di Algoritmi Avanzati sottomesso al prof. Bresolin dell'Università di Padova.
Il progetto richiede di implementare almeno 2 algoritmi che risolvono istanze di : Capacitated Vehicle Routing Problem.
Nel seguente elaborato oltre ad implementare i 2 algoritmi richiesti , il primo come metodo costruttivo; Clarke and Wright e il secondo come metodo a 2 fasi:ClusterFirst - Route Second per il quale sono state implementate due tecniche di routing: Nearest-Neighbourn e Dijkastra.
Sono stati implementati altri 2 algoritmi uno che cade sempre nella classe dei metodi a 2 fasi: RouteFirst - ClusterSecond, e un altro che cade nella categoria degli algoritmi metauristici detti anche algoritmi genetici.
Inoltre, è stato implementato un metodo di selezione dei centroidi che tiene conto sia della distanza dal deposito sia della distanza inter-cluster.
Tutti gli algoritmi sono stati testati sulle 16 istanze fornite dalla libreria TSPLIB95.
I risultati confermano che a fronte di una minore accuratezza di una soluzione rispetto all'ottimo ma una maggiore richiesta di risorse in termini di tempo, possono portare a soluzioni ben approssimate.
In particolare scegliere i centrodi con il metodo RadiusRadar permette di guadagnare qualche decimo a fronte di un tempo inore.
Le soluzioni posso essere migliorate attrverso gli algoritmi genetici che fungono da esploratori di uno spazion locale.

\end{abstract}

\section{}

\end{document}
